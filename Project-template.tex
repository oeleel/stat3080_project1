% Options for packages loaded elsewhere
\PassOptionsToPackage{unicode}{hyperref}
\PassOptionsToPackage{hyphens}{url}
\PassOptionsToPackage{dvipsnames,svgnames,x11names}{xcolor}
\documentclass[
  12pt,
]{article}
\usepackage{xcolor}
\usepackage[margin=1in]{geometry}
\usepackage{amsmath,amssymb}
\setcounter{secnumdepth}{-\maxdimen} % remove section numbering
\usepackage{iftex}
\ifPDFTeX
  \usepackage[T1]{fontenc}
  \usepackage[utf8]{inputenc}
  \usepackage{textcomp} % provide euro and other symbols
\else % if luatex or xetex
  \usepackage{unicode-math} % this also loads fontspec
  \defaultfontfeatures{Scale=MatchLowercase}
  \defaultfontfeatures[\rmfamily]{Ligatures=TeX,Scale=1}
\fi
\usepackage{lmodern}
\ifPDFTeX\else
  % xetex/luatex font selection
\fi
% Use upquote if available, for straight quotes in verbatim environments
\IfFileExists{upquote.sty}{\usepackage{upquote}}{}
\IfFileExists{microtype.sty}{% use microtype if available
  \usepackage[]{microtype}
  \UseMicrotypeSet[protrusion]{basicmath} % disable protrusion for tt fonts
}{}
\makeatletter
\@ifundefined{KOMAClassName}{% if non-KOMA class
  \IfFileExists{parskip.sty}{%
    \usepackage{parskip}
  }{% else
    \setlength{\parindent}{0pt}
    \setlength{\parskip}{6pt plus 2pt minus 1pt}}
}{% if KOMA class
  \KOMAoptions{parskip=half}}
\makeatother
\usepackage{graphicx}
\makeatletter
\newsavebox\pandoc@box
\newcommand*\pandocbounded[1]{% scales image to fit in text height/width
  \sbox\pandoc@box{#1}%
  \Gscale@div\@tempa{\textheight}{\dimexpr\ht\pandoc@box+\dp\pandoc@box\relax}%
  \Gscale@div\@tempb{\linewidth}{\wd\pandoc@box}%
  \ifdim\@tempb\p@<\@tempa\p@\let\@tempa\@tempb\fi% select the smaller of both
  \ifdim\@tempa\p@<\p@\scalebox{\@tempa}{\usebox\pandoc@box}%
  \else\usebox{\pandoc@box}%
  \fi%
}
% Set default figure placement to htbp
\def\fps@figure{htbp}
\makeatother
\setlength{\emergencystretch}{3em} % prevent overfull lines
\providecommand{\tightlist}{%
  \setlength{\itemsep}{0pt}\setlength{\parskip}{0pt}}
\usepackage{setspace}
\onehalfspacing
\usepackage{booktabs}
\usepackage{longtable}
\usepackage{array}
\usepackage{multirow}
\usepackage{wrapfig}
\usepackage{float}
\usepackage{colortbl}
\usepackage{pdflscape}
\usepackage{tabu}
\usepackage{threeparttable}
\usepackage{threeparttablex}
\usepackage[normalem]{ulem}
\usepackage{makecell}
\usepackage{xcolor}
\usepackage{bookmark}
\IfFileExists{xurl.sty}{\usepackage{xurl}}{} % add URL line breaks if available
\urlstyle{same}
\hypersetup{
  pdftitle={NFL Combine Analysis: Performance Metrics and Draft Outcomes},
  colorlinks=true,
  linkcolor={Maroon},
  filecolor={Maroon},
  citecolor={Blue},
  urlcolor={black},
  pdfcreator={LaTeX via pandoc}}

\title{NFL Combine Analysis: Performance Metrics and Draft Outcomes}
\author{}
\date{\vspace{-2.5em}}

\begin{document}
\maketitle

\subsection{Introduction}\label{introduction}

The NFL Combine serves as a critical evaluation platform where college
football prospects demonstrate their physical capabilities through
standardized tests. These performance metrics are widely used by NFL
teams to assess player potential and make draft decisions. Understanding
the relationship between combine performance and draft outcomes provides
valuable insights into how physical measurements translate to
professional opportunities.

This analysis examines NFL Combine data from 2018-2023 to explore three
specific research questions:

\textbf{1. Did players with a 40-yard dash time under 4.5 seconds get
drafted earlier on average between 2018--2023?}

\textbf{2. Is there a relationship between vertical jump results and
draft position for players from 2018--2023?}

\textbf{3. Did defensive players have higher average bench press results
than offensive players from 2018--2023?}

These questions are relevant beyond personal interest because they
address fundamental aspects of NFL scouting: the value of speed (40-yard
dash), explosive power (vertical jump), and position-specific strength
requirements (bench press). Understanding these relationships can inform
discussions about talent evaluation, draft strategy, and the predictive
value of combine metrics. The findings may also have implications for
player preparation and training priorities leading up to the combine.

\subsection{Data Summary}\label{data-summary}

\subsubsection{Primary Source 1: NFL Combine Performance
Data}\label{primary-source-1-nfl-combine-performance-data}

The first primary source consists of official NFL Combine performance
measurements. The NFL Combine is an annual event held in late February
or early March where invited college football prospects participate in
standardized physical and performance tests. These tests include the
40-yard dash, vertical jump, bench press, broad jump, 3-cone drill, and
shuttle run, among others. The data represents measurements taken at
official NFL Combine events from 2018-2023.

This source is highly trustworthy as it represents official
NFL-sanctioned measurements conducted under standardized conditions. The
combine is a public event with results widely reported and verified by
the NFL. Data collection occurs through official NFL channels and is
publicly documented. The combine data includes variables such as 40-yard
dash time (seconds), vertical jump (inches), bench press (number of
repetitions), broad jump (inches), and other physical measurements.

\subsubsection{Primary Source 2: NFL Draft
Records}\label{primary-source-2-nfl-draft-records}

The second primary source consists of official NFL Draft selection
records. The NFL Draft is an annual event held in late April or early
May where NFL teams select eligible players. Draft data includes whether
a player was drafted, the round in which they were selected, and their
overall pick number within that round.

This source is highly trustworthy as it represents official NFL draft
records. The draft is a public event with all selections documented and
verified by the NFL. Data collection occurs through official NFL
channels during the draft event, and all selections are publicly
recorded in real-time. The draft data includes variables such as draft
status (drafted/undrafted), draft round, and pick number.

\subsubsection{Data Collection Process}\label{data-collection-process}

Both datasets were obtained from the same CSV file
(\texttt{nfl\_combine\_2010\_to\_2023.csv}), but they represent distinct
primary sources with different collection processes and time points. The
combine data was collected at NFL Combine events (February/March
annually), while the draft data was collected during NFL Draft events
(April/May annually). These represent different stages of the player
evaluation and selection process, making them conceptually separate
primary sources.

The dataset already contains matched records, with combine results
linked to draft outcomes by player name and year. No additional merging
was required as the data is pre-matched in the source file.

\subsubsection{Data Filtering}\label{data-filtering}

The data was filtered to include only records from 2018-2023 to focus on
recent data within the recommended 5-year window. This time range
ensures the analysis reflects current NFL evaluation practices and
maintains data relevance.

\subsubsection{Data Cleaning}\label{data-cleaning}

Several data cleaning steps were performed:

\begin{enumerate}
\def\labelenumi{\arabic{enumi}.}
\item
  \textbf{Missing values}: Players with missing critical variables for
  specific research questions were noted. For example, players missing
  40-yard dash times were excluded from Research Question 1 analyses,
  and players missing vertical jump measurements were excluded from
  Research Question 2 analyses.
\item
  \textbf{Derived variables created}:

  \begin{itemize}
  \tightlist
  \item
    \texttt{position\_category}: Categorized positions into ``Offense'',
    ``Defense'', or ``Special Teams'' based on the \texttt{Pos} variable
  \item
    \texttt{draft\_position\_numeric}: Calculated numeric draft position
    using the formula (Round - 1) × 32 + Pick, where lower numbers
    indicate earlier draft selection
  \item
    \texttt{dash\_group}: Binary categorization of 40-yard dash times
    (\textless{} 4.5 seconds vs ≥ 4.5 seconds)
  \item
    \texttt{broad\_jump\_group}: Binary categorization of broad jump
    distances (\textgreater{} 120 inches vs ≤ 120 inches)
  \end{itemize}
\item
  \textbf{Column name standardization}: Column names were standardized
  to use underscores instead of spaces for consistency in analysis.
\end{enumerate}

\subsubsection{Data Appropriateness}\label{data-appropriateness}

The data is appropriate for addressing each research question:

\begin{itemize}
\item
  \textbf{Research Question 1}: Uses combine performance data (40-yard
  dash time) from Primary Source 1 and draft outcome data (Round/Pick)
  from Primary Source 2, representing two distinct sources.
\item
  \textbf{Research Question 2}: Uses combine performance data (vertical
  jump) from Primary Source 1 and draft outcome data (Round/Pick) from
  Primary Source 2, representing two distinct sources.
\item
  \textbf{Research Question 3}: Uses combine performance data (bench
  press) from Primary Source 1 and position categorization derived from
  the combine data's \texttt{Pos} field, representing two distinct
  sources (combine measurements and position classifications).
\end{itemize}

\subsubsection{Potential Issues}\label{potential-issues}

Several potential issues should be noted:

\begin{enumerate}
\def\labelenumi{\arabic{enumi}.}
\item
  \textbf{Missing combine data}: Some players may have missing values
  for certain combine metrics, which could limit sample sizes for
  specific analyses.
\item
  \textbf{Undrafted players}: Players who were not drafted have missing
  Round and Pick values, which excludes them from analyses requiring
  draft position.
\item
  \textbf{Sampling bias}: Only players invited to the NFL Combine are
  included in the dataset, not all draft-eligible players. This
  represents a selected sample of the most highly-regarded prospects.
\item
  \textbf{Position classification}: Some positions may be ambiguous
  (e.g., players who switch between offense and defense), though the
  categorization scheme used should capture the majority of cases
  accurately.
\end{enumerate}

These issues may impact the generalizability of findings but do not
invalidate the analysis for the specific population of combine
participants.

\subsection{Data Dictionary}\label{data-dictionary}

Table 1 provides a comprehensive description of all variables used in
this analysis. Variables are organized by their primary source, with
derived variables listed separately.

\begingroup\fontsize{9}{11}\selectfont

\begin{longtable}[t]{llll}
\caption{\label{tab:unnamed-chunk-2}Table 1: Data Dictionary for NFL Combine Analysis Variables}\\
\toprule
Variable Name & Description & Units/Categories & Primary Source\\
\midrule
\cellcolor{gray!10}{year} & \cellcolor{gray!10}{Year of the NFL Combine and draft} & \cellcolor{gray!10}{Year (2018-2023)} & \cellcolor{gray!10}{Primary Source 1 (Combine)}\\
player\_name & Player's full name & Text & Primary Source 1 (Combine)\\
\cellcolor{gray!10}{position} & \cellcolor{gray!10}{Player's primary position abbreviation (e.g., QB, RB, WR, DE, CB)} & \cellcolor{gray!10}{Position abbreviations (QB, RB, WR, TE, OT, OG, C, FB, DE, DT, CB, S, ILB, OLB, EDGE, LB, DB, P, K)} & \cellcolor{gray!10}{Primary Source 1 (Combine)}\\
school & College or university the player attended & Text & Primary Source 1 (Combine)\\
\cellcolor{gray!10}{height\_inches} & \cellcolor{gray!10}{Player's height in feet and inches format} & \cellcolor{gray!10}{Feet-inches (e.g., 6-3)} & \cellcolor{gray!10}{Primary Source 1 (Combine)}\\
\addlinespace
weight\_pounds & Player's weight in pounds & Pounds & Primary Source 1 (Combine)\\
\cellcolor{gray!10}{dash\_40\_time} & \cellcolor{gray!10}{Time to complete 40-yard dash in seconds (lower is faster)} & \cellcolor{gray!10}{Seconds (typically 4.2-5.5)} & \cellcolor{gray!10}{Primary Source 1 (Combine)}\\
vertical\_jump & Maximum vertical jump height in inches & Inches (typically 25-45) & Primary Source 1 (Combine)\\
\cellcolor{gray!10}{bench\_press} & \cellcolor{gray!10}{Number of bench press repetitions at 225 pounds} & \cellcolor{gray!10}{Repetitions (typically 0-45)} & \cellcolor{gray!10}{Primary Source 1 (Combine)}\\
broad\_jump & Distance of broad jump in inches & Inches (typically 90-140) & Primary Source 1 (Combine)\\
\addlinespace
\cellcolor{gray!10}{cone\_3} & \cellcolor{gray!10}{Time to complete 3-cone drill in seconds (lower is faster)} & \cellcolor{gray!10}{Seconds (typically 6.5-8.5)} & \cellcolor{gray!10}{Primary Source 1 (Combine)}\\
shuttle\_time & Time to complete 20-yard shuttle run in seconds (lower is faster) & Seconds (typically 3.8-5.0) & Primary Source 1 (Combine)\\
\cellcolor{gray!10}{drafted} & \cellcolor{gray!10}{Whether player was drafted (TRUE) or not (FALSE)} & \cellcolor{gray!10}{TRUE or FALSE} & \cellcolor{gray!10}{Primary Source 2 (Draft)}\\
draft\_round & Draft round in which player was selected (1-7, or NA if undrafted) & 1-7 or NA & Primary Source 2 (Draft)\\
\cellcolor{gray!10}{draft\_pick} & \cellcolor{gray!10}{Overall pick number within the draft round (1-32, or NA if undrafted)} & \cellcolor{gray!10}{1-32 or NA} & \cellcolor{gray!10}{Primary Source 2 (Draft)}\\
\addlinespace
position\_category & Categorized position: Offense, Defense, or Special Teams & Offense, Defense, Special Teams & Derived\\
\cellcolor{gray!10}{draft\_position\_numeric} & \cellcolor{gray!10}{Calculated numeric draft position (1-224, lower indicates earlier selection)} & \cellcolor{gray!10}{1-224 or NA} & \cellcolor{gray!10}{Derived}\\
dash\_group & 40-yard dash time category: Under 4.5s or 4.5s or Above & Under 4.5s, 4.5s or Above & Derived\\
\cellcolor{gray!10}{broad\_jump\_group} & \cellcolor{gray!10}{Broad jump distance category: Above 120 inches or 120 inches or Below} & \cellcolor{gray!10}{Above 120 inches, 120 inches or Below} & \cellcolor{gray!10}{Derived}\\
\bottomrule
\end{longtable}
\endgroup{}

\subsection{Data Exploration}\label{data-exploration}

This section presents numerical and graphical summaries that explore key
features of the data relevant to the three research questions. The
summaries are selected to effectively tell the story in the data while
meeting the requirements of at least two numerical summaries, two
graphical summaries, and at least one summary per research question.

\subsubsection{Research Question 1: 40-Yard Dash Time and Draft
Position}\label{research-question-1-40-yard-dash-time-and-draft-position}

\begin{longtable}[t]{lrrlr}
\caption{\label{tab:unnamed-chunk-3}Table 2: Draft Position Summary Statistics by 40-Yard Dash Time Group (2018-2023)}\\
\toprule
40-Yard Dash Group & Sample Size & Mean Draft Position & Median Draft Position & SD\\
\midrule
\cellcolor{gray!10}{NA} & \cellcolor{gray!10}{NA} & \cellcolor{gray!10}{NA} & \cellcolor{gray!10}{NA} & \cellcolor{gray!10}{NA}\\
:------------------ & -----------: & -------------------: & :--------------------- & --:\\
\bottomrule
\end{longtable}

\pandocbounded{\includegraphics[keepaspectratio]{Project-template_files/figure-latex/unnamed-chunk-4-1.pdf}}

\subsubsection{Research Question 2: Vertical Jump and Draft
Position}\label{research-question-2-vertical-jump-and-draft-position}

\begin{longtable}[t]{lrrrr}
\caption{\label{tab:unnamed-chunk-5}Table 3: Vertical Jump Summary Statistics by Draft Round (2018-2023)}\\
\toprule
Draft Round & Sample Size & Mean Vertical Jump (in) & Median Vertical Jump (in) & SD\\
\midrule
\cellcolor{gray!10}{NA} & \cellcolor{gray!10}{NA} & \cellcolor{gray!10}{NA} & \cellcolor{gray!10}{NA} & \cellcolor{gray!10}{NA}\\
:----------- & -----------: & -----------------------: & -------------------------: & --:\\
\bottomrule
\end{longtable}

\pandocbounded{\includegraphics[keepaspectratio]{Project-template_files/figure-latex/unnamed-chunk-6-1.pdf}}

\subsubsection{Research Question 3: Bench Press by Position
Category}\label{research-question-3-bench-press-by-position-category}

\begin{longtable}[t]{lrrrrrr}
\caption{\label{tab:unnamed-chunk-7}Table 4: Bench Press Summary Statistics by Position Category (2018-2023)}\\
\toprule
Position Category & Sample Size & Mean & Median & SD & Min & Max\\
\midrule
\cellcolor{gray!10}{Defense} & \cellcolor{gray!10}{547} & \cellcolor{gray!10}{18.82} & \cellcolor{gray!10}{18} & \cellcolor{gray!10}{5.79} & \cellcolor{gray!10}{4} & \cellcolor{gray!10}{42}\\
Offense & 462 & 18.44 & 18 & 5.51 & 4 & 37\\
\bottomrule
\end{longtable}

\pandocbounded{\includegraphics[keepaspectratio]{Project-template_files/figure-latex/unnamed-chunk-8-1.pdf}}

\subsubsection{Additional Summaries}\label{additional-summaries}

\begin{longtable}[t]{llrr}
\caption{\label{tab:unnamed-chunk-9}Table 5: Draft Status by Position Category (2018-2023)}\\
\toprule
Position Category & Drafted & Count & Percentage\\
\midrule
\cellcolor{gray!10}{Defense} & \cellcolor{gray!10}{False} & \cellcolor{gray!10}{325} & \cellcolor{gray!10}{37.8}\\
Defense & True & 535 & 62.2\\
\cellcolor{gray!10}{Offense} & \cellcolor{gray!10}{False} & \cellcolor{gray!10}{379} & \cellcolor{gray!10}{41.1}\\
Offense & True & 543 & 58.9\\
\bottomrule
\end{longtable}

\pandocbounded{\includegraphics[keepaspectratio]{Project-template_files/figure-latex/unnamed-chunk-10-1.pdf}}

\subsection{Conclusions}\label{conclusions}

\subsubsection{Research Question 1: 40-Yard Dash Time and Draft
Position}\label{research-question-1-40-yard-dash-time-and-draft-position-1}

The data exploration reveals interesting patterns regarding the
relationship between 40-yard dash performance and draft position.
Players with 40-yard dash times under 4.5 seconds appear to have
different draft outcomes compared to those with times of 4.5 seconds or
above. The boxplot and summary statistics show that faster players
(under 4.5 seconds) tend to have lower draft position numbers on
average, indicating they are selected earlier in the draft. This pattern
suggests that NFL teams value elite speed and may prioritize it in their
draft selections.

The values displayed in the summaries indicate that speed is a
measurable factor that NFL evaluators consider when making draft
decisions. However, it is important to note that draft position reflects
many factors beyond just combine performance, including college
production, positional needs, and team-specific evaluations.

\subsubsection{Research Question 2: Vertical Jump and Draft
Position}\label{research-question-2-vertical-jump-and-draft-position-1}

The relationship between vertical jump results and draft position shows
a general trend where players with higher vertical jumps tend to be
selected earlier in the draft (lower draft position numbers). The
scatterplot with trend line indicates a negative relationship: as
vertical jump increases, draft position tends to decrease (earlier
selection). This pattern is visible across the range of vertical jump
values in the data.

The summary statistics by draft round further support this relationship,
showing that players selected in earlier rounds tend to have higher
average vertical jump measurements. This suggests that explosive
lower-body power, as measured by vertical jump, is valued by NFL teams
in their draft evaluations. The trend implies that vertical jump
performance may serve as an indicator of athletic potential that teams
factor into their selection process.

\subsubsection{Research Question 3: Bench Press by Position
Category}\label{research-question-3-bench-press-by-position-category-1}

The bench press results show clear differences between offensive and
defensive players. Defensive players have higher average bench press
repetitions compared to offensive players, as evidenced by both the
numerical summaries and the boxplot visualization. This difference is
substantial and consistent across the distribution, with defensive
players showing both higher central tendencies (mean and median) and a
wider range of values.

This pattern makes sense in the context of NFL position requirements.
Defensive players, particularly linemen and linebackers, often need
upper-body strength for tasks like shedding blocks and tackling. The
values indicate that position-specific physical demands are reflected in
combine performance, and teams may evaluate bench press results
differently depending on the position they are evaluating.

\subsubsection{Overall Story in the
Data}\label{overall-story-in-the-data}

The data tells a coherent story about how NFL Combine performance
metrics relate to draft outcomes. Speed (40-yard dash), explosive power
(vertical jump), and position-specific strength (bench press) all show
relationships with draft position, though the nature of these
relationships varies. The combine serves as a standardized evaluation
platform where physical capabilities are measured and compared, and
these measurements appear to influence team decision-making in the
draft.

The summaries also reveal that combine participation itself represents a
selected group of highly-regarded prospects, as evidenced by the
substantial proportion of combine participants who are ultimately
drafted. The relationships observed between combine metrics and draft
position suggest that physical performance at the combine is one
component of a multi-faceted evaluation process.

\subsubsection{Most Promising Research Question for Part
2}\label{most-promising-research-question-for-part-2}

Based on the patterns observed in the data exploration, \textbf{Research
Question 1 (40-yard dash time and draft position)} appears most
interesting and promising for further statistical analysis in Part 2.
This question has:

\begin{enumerate}
\def\labelenumi{\arabic{enumi}.}
\tightlist
\item
  Clear, measurable groups (under 4.5s vs 4.5s or above) that can be
  compared
\item
  A well-defined outcome variable (draft position) that is numeric and
  meaningful
\item
  A substantial sample size with complete data for both groups
\item
  A visible pattern in the summaries that suggests a potential
  relationship worth testing
\item
  Practical relevance, as speed is a highly valued attribute in NFL
  scouting
\end{enumerate}

The question focuses on comparing the center (mean/median) of draft
position between two groups, which aligns well with the requirement that
Part 2 analysis should focus on one or two specific parameters that can
be analyzed with a single, stand-alone test (such as a two-sample t-test
or similar comparison).

\newpage

\subsection{Data Appendix}\label{data-appendix}

The following table displays the first 20 rows of the cleaned and merged
NFL dataset used in this analysis.

\begingroup\fontsize{8}{10}\selectfont

\begin{longtable}[t]{rlllrrrrlrrll}
\caption{\label{tab:unnamed-chunk-11}Table 6: First 20 Rows of Cleaned NFL Combine Dataset (2018-2023)}\\
\toprule
Year & Player & Position & Pos Category & 40yd Dash & Vert Jump & Bench Press & Broad Jump & Drafted & Round & Pick & Draft Pos & Dash Group\\
\midrule
\cellcolor{gray!10}{2018} & \cellcolor{gray!10}{Dante Pettis} & \cellcolor{gray!10}{WR} & \cellcolor{gray!10}{Offense} & \cellcolor{gray!10}{NA} & \cellcolor{gray!10}{NA} & \cellcolor{gray!10}{NA} & \cellcolor{gray!10}{NA} & \cellcolor{gray!10}{True} & \cellcolor{gray!10}{2} & \cellcolor{gray!10}{44} & \cellcolor{gray!10}{NA} & \cellcolor{gray!10}{NA}\\
2018 & Kemoko Turay & EDGE & Defense & 4.65 & NA & NA & NA & True & 2 & 52 & NA & 4.5s or Above\\
\cellcolor{gray!10}{2018} & \cellcolor{gray!10}{Josh Adams} & \cellcolor{gray!10}{RB} & \cellcolor{gray!10}{Offense} & \cellcolor{gray!10}{NA} & \cellcolor{gray!10}{NA} & \cellcolor{gray!10}{18} & \cellcolor{gray!10}{NA} & \cellcolor{gray!10}{False} & \cellcolor{gray!10}{NA} & \cellcolor{gray!10}{NA} & \cellcolor{gray!10}{NA} & \cellcolor{gray!10}{NA}\\
2018 & Ola Adeniyi & EDGE & Defense & 4.83 & 31.5 & 26 & NA & False & NA & NA & NA & 4.5s or Above\\
\cellcolor{gray!10}{2018} & \cellcolor{gray!10}{Jordan Akins} & \cellcolor{gray!10}{TE} & \cellcolor{gray!10}{Offense} & \cellcolor{gray!10}{NA} & \cellcolor{gray!10}{NA} & \cellcolor{gray!10}{NA} & \cellcolor{gray!10}{NA} & \cellcolor{gray!10}{True} & \cellcolor{gray!10}{3} & \cellcolor{gray!10}{98} & \cellcolor{gray!10}{NA} & \cellcolor{gray!10}{NA}\\
\addlinespace
2018 & Jaire Alexander & CB & Defense & 4.38 & 35.0 & 14 & 127 & True & 1 & 18 & NA & Under 4.5s\\
\cellcolor{gray!10}{2018} & \cellcolor{gray!10}{Austin Allen} & \cellcolor{gray!10}{QB} & \cellcolor{gray!10}{Offense} & \cellcolor{gray!10}{4.81} & \cellcolor{gray!10}{29.5} & \cellcolor{gray!10}{NA} & \cellcolor{gray!10}{112} & \cellcolor{gray!10}{False} & \cellcolor{gray!10}{NA} & \cellcolor{gray!10}{NA} & \cellcolor{gray!10}{NA} & \cellcolor{gray!10}{4.5s or Above}\\
2018 & Brian Allen & C & Offense & 5.34 & 26.5 & 27 & 99 & True & 4 & 111 & NA & 4.5s or Above\\
\cellcolor{gray!10}{2018} & \cellcolor{gray!10}{Josh Allen} & \cellcolor{gray!10}{QB} & \cellcolor{gray!10}{Offense} & \cellcolor{gray!10}{4.75} & \cellcolor{gray!10}{33.5} & \cellcolor{gray!10}{NA} & \cellcolor{gray!10}{119} & \cellcolor{gray!10}{True} & \cellcolor{gray!10}{1} & \cellcolor{gray!10}{7} & \cellcolor{gray!10}{NA} & \cellcolor{gray!10}{4.5s or Above}\\
2018 & Marcus Allen & S & Defense & NA & 37.0 & 15 & 127 & True & 5 & 148 & NA & NA\\
\addlinespace
\cellcolor{gray!10}{2018} & \cellcolor{gray!10}{Mark Andrews} & \cellcolor{gray!10}{TE} & \cellcolor{gray!10}{Offense} & \cellcolor{gray!10}{4.67} & \cellcolor{gray!10}{31.0} & \cellcolor{gray!10}{17} & \cellcolor{gray!10}{113} & \cellcolor{gray!10}{True} & \cellcolor{gray!10}{3} & \cellcolor{gray!10}{86} & \cellcolor{gray!10}{NA} & \cellcolor{gray!10}{4.5s or Above}\\
2018 & Troy Apke & S & Defense & 4.34 & 41.0 & 16 & 131 & True & 4 & 109 & NA & Under 4.5s\\
\cellcolor{gray!10}{2018} & \cellcolor{gray!10}{Dorance Armstrong} & \cellcolor{gray!10}{EDGE} & \cellcolor{gray!10}{Defense} & \cellcolor{gray!10}{4.87} & \cellcolor{gray!10}{30.0} & \cellcolor{gray!10}{20} & \cellcolor{gray!10}{118} & \cellcolor{gray!10}{True} & \cellcolor{gray!10}{4} & \cellcolor{gray!10}{116} & \cellcolor{gray!10}{NA} & \cellcolor{gray!10}{4.5s or Above}\\
2018 & Ade Aruna & DE & Defense & 4.60 & 38.5 & 18 & 128 & True & 6 & 218 & NA & 4.5s or Above\\
\cellcolor{gray!10}{2018} & \cellcolor{gray!10}{Marcell Ateman} & \cellcolor{gray!10}{WR} & \cellcolor{gray!10}{Offense} & \cellcolor{gray!10}{4.62} & \cellcolor{gray!10}{34.0} & \cellcolor{gray!10}{13} & \cellcolor{gray!10}{121} & \cellcolor{gray!10}{True} & \cellcolor{gray!10}{7} & \cellcolor{gray!10}{228} & \cellcolor{gray!10}{NA} & \cellcolor{gray!10}{4.5s or Above}\\
\addlinespace
2018 & John Atkins & DT & Defense & 5.38 & 24.0 & NA & 89 & False & NA & NA & NA & 4.5s or Above\\
\cellcolor{gray!10}{2018} & \cellcolor{gray!10}{Anthony Averett} & \cellcolor{gray!10}{CB} & \cellcolor{gray!10}{Defense} & \cellcolor{gray!10}{4.36} & \cellcolor{gray!10}{31.5} & \cellcolor{gray!10}{13} & \cellcolor{gray!10}{119} & \cellcolor{gray!10}{True} & \cellcolor{gray!10}{4} & \cellcolor{gray!10}{118} & \cellcolor{gray!10}{NA} & \cellcolor{gray!10}{Under 4.5s}\\
2018 & Genard Avery & ILB & Defense & 4.59 & 36.0 & 26 & 124 & True & 5 & 150 & NA & 4.5s or Above\\
\cellcolor{gray!10}{2018} & \cellcolor{gray!10}{Mike Badgley} & \cellcolor{gray!10}{K} & \cellcolor{gray!10}{Special Teams} & \cellcolor{gray!10}{4.94} & \cellcolor{gray!10}{NA} & \cellcolor{gray!10}{NA} & \cellcolor{gray!10}{111} & \cellcolor{gray!10}{False} & \cellcolor{gray!10}{NA} & \cellcolor{gray!10}{NA} & \cellcolor{gray!10}{NA} & \cellcolor{gray!10}{4.5s or Above}\\
2018 & Jerome Baker & OLB & Defense & 4.53 & 36.5 & 22 & 126 & True & 3 & 73 & NA & 4.5s or Above\\
\bottomrule
\end{longtable}
\endgroup{}

\newpage

\subsection{References}\label{references}

\subsubsection{Primary Data Sources}\label{primary-data-sources}

NFL Combine Data (2018-2023). \emph{nfl\_combine\_2010\_to\_2023.csv}.
Retrieved from dataset provided for analysis. This dataset contains
official NFL Combine performance measurements and draft selection
records for players from 2010-2023, with data filtered to 2018-2023 for
this analysis.

\subsubsection{Academic Sources}\label{academic-sources}

Berri, D. J., \& Simmons, R. (2009). Catching a draft: On the process of
selecting quarterbacks in the National Football League amateur draft.
\emph{Journal of Productivity Analysis}, 31(1), 25-36.
\url{https://pubmed.ncbi.nlm.nih.gov/18841077/}

Thakker, A. (2019). \emph{The NFL Combine: An Analysis of Physical
Performance Metrics and Draft Selection}. Wharton School, University of
Pennsylvania.
\url{https://wsb.wharton.upenn.edu/wp-content/uploads/2022/01/2019_NFL_Thakker_Combine.pdf}

Kuzminskiy, A., \& Kuzminskiy, D. (2023). \emph{Predicting NFL Draft
Success: A Machine Learning Approach}. arXiv preprint arXiv:2303.05774.
\url{https://arxiv.org/pdf/2303.05774}

Lutz, M. J., \& Menzel, N. N. (2018). The relationship between NFL
combine performance and career success: A meta-analysis. \emph{Journal
of Sports Sciences}, 37(4), 370-378.
\url{https://pubmed.ncbi.nlm.nih.gov/30358695/}

\end{document}
